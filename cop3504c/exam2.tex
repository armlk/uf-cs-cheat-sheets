\documentclass[10pt,a4paper]{article}

% PACKAGES
\usepackage[utf8]{inputenc}
\usepackage{amsmath}
\usepackage{xcolor}
\usepackage{geometry}
\usepackage{multicol} % For two-column layout
\usepackage{listings} % For code blocks
\usepackage{sectsty} % To custom-format section headings
\usepackage{enumitem} % For compact lists

% PAGE GEOMETRY (TIGHT MARGINS)
\geometry{
  a4paper,
  total={180mm,270mm},
  left=5mm,
  right = 5mm,
  top=5mm,
  bottom = 5mm
}

% TITLE
\title{\textbf{C++, GDB, \& Valgrind Cheat Sheet}}
\author{} % No author
\date{} % No date

% CUSTOM SECTION FORMATTING
\sectionfont{\fontsize{12}{15}\selectfont\textbf}
\subsectionfont{\fontsize{10}{12}\selectfont\textit}

% CUSTOM LISTINGS (CODE) STYLE
\definecolor{codegreen}{rgb}{0,0.6,0}
\definecolor{codepurple}{rgb}{0.58,0,0.82}
\definecolor{backcolour}{rgb}{0.97,0.97,0.97}
\definecolor{codegray}{rgb}{0.5,0.5,0.5}

\lstdefinestyle{cppstyle}{
  backgroundcolor=\color{backcolour},
  commentstyle=\color{codegreen},
  keywordstyle=\color{blue},
  stringstyle=\color{codepurple},
  basicstyle=\ttfamily\footnotesize,
  breakatwhitespace=false,
  breaklines=true,
  captionpos=b,
  keepspaces=true,
  showspaces=false,
  showstringspaces=false,
  showtabs=false,
  tabsize=2,
  language=C++
}
\lstset{style=cppstyle} % Set C++ style as default

% COMPACT LISTS
\setlist[itemize]{noitemsep, nolistsep, topsep=2pt, partopsep=0pt}
\setlist[enumerate]{noitemsep, nolistsep, topsep=2pt, partopsep=0pt}

% NO PAGE NUMBERS
\pagestyle{empty}

% DOCUMENT START
\begin{document}

\begin{multicols}{2} % Start two-column layout
  \subsection*{Access Specifiers}
  Determines how base class members are inherited:

  \begin{itemize}
    \item \textbf{\texttt{public} inheritance}: (Most common)
      \begin{itemize}
        \item Base \texttt{public} $\to$ Derived \texttt{public}
        \item Base \texttt{protected} $\to$ Derived \texttt{protected}
        \item Base \texttt{private} $\to$ Inaccessible
      \end{itemize}
    \item \textbf{\texttt{protected} inheritance}:
      \begin{itemize}
        \item Base \texttt{public} $\to$ Derived \texttt{protected}
        \item Base \texttt{protected} $\to$ Derived \texttt{protected}
      \end{itemize}
    \item \textbf{\texttt{private} inheritance}:
      \begin{itemize}
        \item Base \texttt{public} $\to$ Derived \texttt{private}
        \item Base \texttt{protected} $\to$ Derived \texttt{private}
      \end{itemize}
  \end{itemize}

  \subsection*{Constructor / Destructor Order}
  \begin{itemize}
    \item \textbf{Construction}: Base class constructor runs
      \textit{first}, then derived class constructor.
    \item \textbf{Destruction}: Derived class destructor runs
      \textit{first}, then base class destructor (reverse order).
  \end{itemize}

  \subsection*{Calling Base Constructor}
  Use the member initializer list to pass arguments to the base constructor.
\begin{lstlisting}
class Base {
public:
    Base(int x) { /* ... */ }
};

class Derived : public Base {
public:
    Derived(int x, int y) : Base(x) {
        // y is used by Derived
    }
};
\end{lstlisting}

  % --- C++ POLYMORPHISM ---
  \subsection*{Virtual Functions}
  A function in the base class declared with \texttt{virtual}. The
  dynamic type (the actual object) determines which function version
  is called at runtime, not the static type (the pointer).

\begin{lstlisting}
class Animal {
public:
    virtual void speak() {
        std::cout << "???";
    }
    // ALWAYS make base class destructors virtual!
    virtual ~Animal() {}
};

class Dog : public Animal {
public:
    // 'override' is optional but recommended: compiler checks that it actually overrides a base virtual function.
    virtual void speak() override {
        std::cout << "Woof!";
    }
};
\end{lstlisting}
  \textbf{Usage:}
\begin{lstlisting}
Animal* p = new Dog();
p->speak(); // Calls Dog::speak()
delete p;   // Calls Dog::~Dog(), then
            // Animal::~Animal()
\end{lstlisting}

  \subsection*{Abstract Classes \& Pure Virtual}
  A class with one or more \textbf{pure virtual functions} (\texttt{=
  0}) is an \textbf{abstract class} and cannot be instantiated. It
  \textit{must} be subclassed.
\begin{lstlisting}
class Shape { // Abstract Base Class
public:
    // Pure virtual function
    virtual double getArea() = 0;
    virtual ~Shape() {}
};

class Circle : public Shape {
private:
    double r;
public:
    // Must implement all pure
    // virtual functions
    virtual double getArea() override {
        return 3.14 * r * r;
    }
};
\end{lstlisting}

  % --- GDB ---
  \section*{GDB (GNU Debugger)}

  Compile with the \texttt{-g} flag to include debug symbols.

  \subsection*{Starting}
  \begin{itemize}
    \item \texttt{gdb ./my\_program} : Start GDB session.
    \item \texttt{gdb -tui ./my\_program} : Start with text-based GUI
  \end{itemize}

  \subsection*{Essential Commands}
  \begin{itemize}
    \item \texttt{run (r)}: Start program (with optional args, e.g.,
      \texttt{r arg1}).
    \item \texttt{break (b) <loc>}: Set breakpoint.
      \begin{itemize}
      \end{itemize}
    \item \texttt{continue (c)}: Continue execution to next breakpoint.
    \item \texttt{next (n)}: Step over (executes line, doesn't enter funcs).
    \item \texttt{step (s)}: Step into (enters function calls).
    \item \texttt{finish}: Step out (run until current function returns).
    \item \texttt{print (p) <var>}: Print value of a variable.
    \item \texttt{backtrace (bt)}: Show the call stack (func call history).
    \item \texttt{list (l)}: Show source code around current line.
    \item \texttt{info break}: List all breakpoints.
    \item \texttt{delete <num>}: Delete breakpoint by number.
    \item \texttt{quit (q)}: Exit GDB.
  \end{itemize}

  % --- VALGRIND ---
  \section*{Valgrind (Memcheck)}

  A tool for detecting memory errors (leaks, invalid access, etc.).
  Compile with \texttt{-g} for useful line numbers.

  The default tool is \textbf{Memcheck}.
\begin{lstlisting}[language=Bash, style=]
valgrind [options] ./my_program
\end{lstlisting}

  \subsection*{Most Common Flags}
  \begin{itemize}
    \item \texttt{--leak-check=full}: (Essential) Shows details for
    \item \texttt{--log-file="report.txt"}: Write output to a file.
  \end{itemize}

  \subsection*{Interpreting Output}
  Look for:
  \begin{itemize}
    \item \textbf{Invalid read / Invalid write}: Accessing memory you
      don't own (e.g., array out of bounds, using a dangling pointer).
    \item \textbf{Conditional jump... depends on uninitialised
      value(s)}: Using a variable (e.g., in an \texttt{if} statement)
      before it was given a value.
    \item \textbf{Mismatched free() / delete / delete[]}: Using
      \texttt{delete} on memory from \texttt{malloc}, or
      \texttt{delete[]} on \texttt{new}, or \texttt{delete} on \texttt{new[]}.
    \item \textbf{HEAP SUMMARY}:
      \begin{itemize}
        \item \texttt{in use at exit}: Should be 0 bytes.
        \item \texttt{definitely lost}: True memory leaks. You lost
          all pointers to this memory. \textbf{This is the one to fix!}
      \end{itemize}
  \end{itemize}

\end{multicols} % End two-column layout
\end{document}
