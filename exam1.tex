\documentclass{article}
\usepackage[letterpaper, margin=0.2in]{geometry}
\usepackage{multicol}
\usepackage{amsmath}
\usepackage{amssymb}
\setlength{\columnsep}{1cm}
\setlength{\columnseprule}{1pt}

\newcommand{\logicLaw}[3]{
    \noindent\parbox{\linewidth}{
        #1\hfill #3\\
        #2
        \par
        \vspace{.5\baselineskip}
        \hrule
        \vspace{.5\baselineskip}
    }
}

\newcommand{\inferenceRule}[5]{
    \noindent\parbox{\linewidth}{
        #1\hfill#4\hfill#5\\
        #2\\
        \rule[5pt]{1.5cm}{1pt}
        \vspace{-.6\baselineskip}

        \noindent$\therefore$ #3
        \par
        \vspace{.5\baselineskip}
        \hrule
        \vspace{.5\baselineskip}
    }
}

\newcommand{\quantifierStatement}[4]{
    \noindent\parbox{\linewidth}{
        #1\hfill#3\hfill#4\\
        #2
        \par
        \vspace{.5\baselineskip}
        \hrule
        \vspace{.5\baselineskip}
    }
}

\newcommand{\twoPartStatement}[2]{
    \noindent\parbox{\linewidth}{
        #1\hfill#2
        \par
        \vspace{.5\baselineskip}
        \hrule
        \vspace{.5\baselineskip}
    }
}

\newcommand{\twoFields}[2]{
    \noindent
    \textbf{#1}\hfill\textbf{#2} \\
    \par
}

\newcommand{\threeFields}[3]{
    \noindent
    \textbf{#1}\hfill\textbf{#2}\hfill\textbf{#3}\\
    \par
}

\newcommand{\sectionTitle}[1]{
    \begin{center}
        \large #1
    \end{center}
}

\begin{document}

\sectionTitle{logic laws (logical equivalences)}
\begin{multicols}{2}
    \raggedcolumns
    \twoFields{equivalence}{name}
    \logicLaw
        {$p\land T\equiv p$}
        {$p\lor F\equiv p$}
        {identity laws}
    \logicLaw
        {$p\lor T\equiv T$}
        {$p\land F\equiv F$}
        {domination laws}
    \logicLaw
        {$p\lor p\equiv p$}
        {$p\land p\equiv p$}
        {idempotent laws}
    \logicLaw
        {$\neg(\neg p)\equiv p$}
        {}
        {double negation law}
    \logicLaw
        {$p\lor q\equiv q\lor p$}
        {$p\land q\equiv q\land p$}
        {commutative laws}
    \logicLaw
        {$(p\lor q)\lor r\equiv p\lor (q\lor r)$}
        {$(p\land q)\land r\equiv p\land (q\land r)$}
        {associative laws}
    \twoFields{equivalence}{name}
    \logicLaw
        {$p\lor (q\land r)\equiv (p\lor q)\land (p\lor r)$}
        {$p\land (q\lor r)\equiv (p\land q)\lor (p\land r)$}
        {distributive laws}
    \logicLaw
        {$\neg(p\land q)\equiv\neg p\lor\neg q$}
        {$\neg(p\lor q)\equiv\neg p\land\neg q$}
        {De Morgan's laws}
    \logicLaw
        {$p\lor(p\land q)\equiv p$}
        {$p\land(p\lor q)\equiv p$}
        {absorption}
    \logicLaw
        {$p\lor\neg p\equiv T$}
        {}
        {tautology}
    \logicLaw
        {$p\land\neg p\equiv F$}
        {}
        {contradiction}
    \logicLaw
        {$(p\rightarrow q)\equiv(\neg p\lor q)$}
        {}
        {implication equivalence}
\end{multicols}
\vspace{-1.5\baselineskip}

\sectionTitle{rules of inference}
\begin{multicols}{2}
    \raggedcolumns
    \threeFields{rule}{tautology}{name}
    \inferenceRule
        {$p$}
        {$p\rightarrow q$}
        {$q$}
        {$[p\land(p\rightarrow q)]\rightarrow q$}
        {modus ponens}
    \inferenceRule
        {$\neg q$}
        {$p\rightarrow q$}
        {$\neg p$}
        {$[\neg q\land(p\rightarrow q)]\rightarrow \neg p$}
        {modus tollens}
    \inferenceRule
        {$p\rightarrow q$}
        {$q\rightarrow r$}
        {$p\rightarrow r$}
        {$[(p\rightarrow q)\land(q\rightarrow r)]\rightarrow(p\rightarrow r)$}
        {hypothetical syllogism}
    \inferenceRule
        {$p\lor q$}
        {$\neg p$}
        {$q$}
        {$[(p\lor q)\land\neg p]\rightarrow q$}
        {disjunctive syllogism}
    \inferenceRule
        {}
        {$p$}
        {$p\lor q$}
        {$p\rightarrow(p\lor q)$}
        {addition}
    \inferenceRule
        {}
        {$p\land q$}
        {$p$}
        {$(p\land q)\rightarrow p$}
        {simplification}
    \threeFields{rule}{tautology}{name}
    \inferenceRule
        {$p$}
        {$q$}
        {$p\land q$}
        {$[(p)\land(q)]\rightarrow(p\land q)$}
        {conjunction}
    \inferenceRule
        {$p\lor q$}
        {$\neg p\lor r$}
        {$q\lor r$}
        {$[(p\lor q)\land(\neg p\lor r)]\rightarrow(q\lor r)$}
        {resolution}
    \inferenceRule
        {}
        {$\forall xP(x)$}
        {$P(c)$}
        {}
        {universal instantiation}
    \inferenceRule
        {}
        {$P(c)$ for an arbitrary $c$}
        {$\forall xP(x)$}
        {}
        {universal generalization}
    \inferenceRule
        {}
        {$\exists xP(x)$}
        {$P(c)$ for some $c$}
        {}
        {existential instantiation}
    \inferenceRule
        {}
        {$P(c)$ for some $c$}
        {$\exists xP(x)$}
        {}
        {existential generalization}
\end{multicols}
\vspace{-1.5\baselineskip}

\sectionTitle{fallacies}
\begin{multicols}{2}
    \raggedcolumns
    \threeFields{rule}{contingency}{name}
    \inferenceRule
        {$p\rightarrow q$}
        {$q$}
        {$p$}
        {$[(p\rightarrow q)\land q]\rightarrow p$}
        {affirming the conclusion}
    \inferenceRule
        {$p\rightarrow q$}
        {$\neg p$}
        {$\neg q$}
        {$[(p\rightarrow q)\land\neg p]\rightarrow\neg q$}
        {denying the hypothesis}
    \threeFields{rule}{contingency}{name}
    \inferenceRule
        {}
        {$p$}
        {$p$}
        {$p\rightarrow p$}
        {begging the question}
\end{multicols}

\sectionTitle{quantifiers}
\threeFields{statement}{when T (true)?}{when F (false)?}
\quantifierStatement
    {$\forall xP(x)$}
    {}
    {P(x) is T for every x}
    {P(x) is F for at least one x}
\quantifierStatement
    {$\exists xP(x)$}
    {}
    {P(x) is T for at least one x}
    {P(x) is F for every x}
\quantifierStatement
    {$\forall x\forall yP(x,y)$}
    {$\forall y\forall xP(x,y)$}
    {$P(x,y)$ is T for every pair x, y}
    {$P(x,y)$ is F for at least one pair x, y}
\quantifierStatement
    {$\forall x\exists yP(x,y)$}
    {}
    {for every x, $P(x,y)$ is T for at least one y}
    {there is at least one x such that $P(x,y)$ is F for every y}
\quantifierStatement
    {$\exists y\forall xP(x,y)$}
    {}
    {for at least one x, $P(x,y)$ is T for every y}
    {for every x, $P(x,y)$ is F for at least one y}
\quantifierStatement
    {$\exists x\exists yP(x,y)$}
    {$\exists y\exists xP(x,y)$}
    {$P(x,y)$ is T for at least one pair x, y}
    {$P(x,y)$ is F for every pair x, y}
\vspace{-\baselineskip}

\sectionTitle{types of proofs}
\twoFields{type}{description}
\twoPartStatement
    {direct}
    {show that if $p$, then $q$ must follow ($p\rightarrow q$)}
\twoPartStatement
    {contrapositive}
    {show that if $\neg q$, then $\neg p$ must follow ($\neg q\rightarrow\neg p$)}
\twoPartStatement
    {vacuous}
    {show that $\neg p$ always holds, making $p\rightarrow q$ a tautology}
\twoPartStatement
    {trivial}
    {show that $q$ always holds, making $p\rightarrow q$ a tautology}
\twoPartStatement
    {contradiction}
    {show that $\neg p$ leads to a contradiction}
\twoPartStatement
    {cases}
    {split the hypothesis into cases and show that $p_i\rightarrow q$ for each case}
\twoPartStatement
    {exhaustive}
    {check a list of all possible cases, with each case representing a single example}
\twoPartStatement
    {existence (constructive)}
    {find an element $a$ (called the witness) such that P(a) is true}
\twoPartStatement
    {existence (nonconstructive)}
    {show $\exists xP(x)$ is true without finding a witness (e.g., by showing $\neg\exists xP(x)$ leads to a contradiction)}
\twoPartStatement
    {uniqueness}
    {show that there is exactly one element with a particular property (show existence and uniqueness)}
\vspace{-\baselineskip}

\sectionTitle{definitions}
\begin{multicols}{2}
    \raggedcolumns
    \twoFields{term}{definition}
    \twoPartStatement
        {hypothesis/conclusion}
        {in $p\rightarrow q$, $p$ is the hypothesis and $q$ is the conclusion}
    \twoPartStatement
        {converse}
        {the converse of $p\rightarrow q$ is $q\rightarrow p$}
    \twoPartStatement
        {contrapositive}
        {the contrapositive of $p\rightarrow q$ is $\neg q\rightarrow\neg p$}
    \twoPartStatement
        {inverse}
        {the inverse of $p\rightarrow q$ is $\neg p\rightarrow\neg q$}
    \twoPartStatement
        {tautology}
        {a proposition that is always true}
    \twoPartStatement
        {contradiction}
        {a proposition that is always false}
    \twoPartStatement
        {contingency}
        {a proposition that is neither a tautology nor a contradiction}
    \twoFields{term}{definition}
    \twoPartStatement
        {corollary}
        {a theorem that can be established directly from a theorem that has been proved}
    \twoPartStatement
        {even number}
        {an integer that can be expressed in the form $n=2k,k\in\mathbb{Z}$}
    \twoPartStatement
        {odd number}
        {an integer than can be expressed in the form $n=2k+1,k\in\mathbb{Z}$}
    \twoPartStatement
        {prime number}
        {an integer that is only divisible by itself and 1; $\forall x((x|n)\rightarrow ((x=n)\lor(x=1)))$}
    \twoPartStatement
        {composite number}
        {an integer that has an integer divisor other than itself and 1; $\exists x((x|n)\land(x\neq n)\land(x\neq 1))$}
\end{multicols}

\end{document}
